\documentclass{article}
\usepackage[utf8]{inputenc}
\usepackage[headings]{fullpage}
\usepackage{algorithm2e, amsfonts, amsmath, amssymb, enumitem, fancyhdr, hyperref, mathtools, minted, MnSymbol, stmaryrd, tikz, xfrac}
\usetikzlibrary{positioning}
\graphicspath{ {.} }

\usepackage[letterpaper,
            left=2cm,
            right=2cm,
            top=3cm,
            bottom=2cm]{geometry}

\def\Name{Tyler Hou}
\def\StudentID{3033060357}
\def\Class{CS 294}
\def\Semester{Fa 2023}
\def\HomeworkNumber{2}
\def\DocumentTitle{\Class{}, \Semester{} -- HW \HomeworkNumber{}}

\newcommand{\Continued}{
    \begin{flushright}
    $\xrightarrow{\text{Continued on next page}}$
    \end{flushright}
    \newpage{}
}
\DeclarePairedDelimiter\ceil{\lceil}{\rceil}
\DeclarePairedDelimiter\floor{\lfloor}{\rfloor}
\DeclarePairedDelimiter\abs{\lvert}{\rvert}
\DeclarePairedDelimiter\linvec{\langle}{\rangle}

\def\Algorithm{\medbreak\textit{Algorithm}}
\def\Proof{\medbreak\textit{Proof of correctness}}
\def\Runtime{\medbreak\textit{Runtime analysis}}

\newcommand{\BigO}{\mathcal{O}}
\newcommand{\qed}{\hfill$\square$}

\title{\vspace{-2cm}\DocumentTitle{}}
\author{\Name{} (SID \StudentID{})}
\date{}

\pagestyle{fancy}
\fancyhf{}
\lhead{\DocumentTitle{} -- \Name{} (\StudentID{})}
\rhead{\thepage}

\begin{document}

\maketitle

\noindent\textbf{Collaborators}:
Chris Douglas,
Vivian Fang,
Altaan Han,
Shadaj Laddad,
Gabriel Matute,
Jiwon Park, \newline
Conor Power,
Ricardo Sandoval

\newcommand{\AND}{\wedge}
\newcommand{\OR}{\vee}
\newcommand{\Vars}{\text{Vars}}

\section{Problem 1}

I'm completely lost on this problem. I tried writing a counterexample checker,
which enumerates over all possible schemas, tries randomly generated (boolean)
tables, and compares the result of $(A \bowtie B) \ltimes Q$ and $(A \ltimes Q)
\bowtie (B \ltimes Q)$.
\url{https://gist.github.com/tylerhou/f308121e88817b46562149c7501ef164}.

The checker cannot find counterexamples when $A$ and $B$ don't separately share
variables with $Q$; i.e. the above equality holds if

$$\Vars(A) \cap \Vars(Q) = \Vars(B) \cap \Vars(Q)$$.

But this feels too strong: since $Q$ contains all variables, this implies that
$\Vars(A) = \Vars(B)$.

The simplest such counterexample when the above does not hold is the database
$A(aq) = \{(0,) (1,)\}$, $B(bq) = \{(0,), (1,)\}$, and $Q(aq, bq) = \{(0, 1),
(1, 0), (1, 1)\}$. Then:

\begin{align*}
    (A \bowtie B) \ltimes Q &= \{(0, 0), (0, 1), (1, 0), (1, 1)\} \ltimes Q           \\
                            &= \{(0, 1), (1, 0), (1, 1)\} = Q                         \\
                            &\neq \{(0, 0), (0, 1) (1, 0), (1, 1)\}                   \\
                            &= \{(0,), (1,)\}_{(aq,)} \bowtie \{(0,), (1,)\}_{(bq,)}  \\
                            &= (A \ltimes Q) \bowtie (B \ltimes Q)                    \\
\end{align*}

\section{Problem 2}

\subsection{Part A}

\begin{enumerate}[label = {\arabic*)}]
    \item $Q_1$ Cyclic. Triangle join shown in lecture.
    \item $Q_2$ Acyclic.

        \begin{align*}
            Q_2 &= R(X, Y, Z) \AND S(Y, Z, U) \AND T(Z, U, V) \\
                &= R(Y, Z) \AND S(Y, Z, U) \AND (T(Z, U) \tag{Remove isolated variables $X$ and $V$} \\
                &= S(Y, Z, U) \tag{Remove ears $R(Y, Z)$ and $T(Z, U)$} \\
                &= - \tag{Remove isolated $Y, Z, U$}
        \end{align*}

    \item $Q_3$ Acyclic.

        \begin{align*}
            Q_3 &= A(X, Y, Z) \AND R(X, Y) \AND S(Y, Z) \AND T(Z, X) \\
                &= A(X, Y, Z) \tag{Remove ears ($R, S, T$ contained in $A$)} \\
                &= - \tag{Remove isolated}
        \end{align*}

    \item $Q_4$ There are no isolated variables; we can only remove ears $A, B,
        C$. But this gives us $Q_1$, which is cyclic.
\end{enumerate}

\subsection{Part B}

\begin{enumerate}
    \item 3. Number a cycle's vertices $1 \to 2 \to 3 \to \ldots \to n \to 1$.
        Then $(0, 1, n) \to (1, 2, n) \to (2, 3, n) \to \ldots \to (n-2, n-1,
        n)$ is a tree decomposition with tree width $3$.
    \item Julia claims that it has tree width $n$. I have no idea how to prove.
    \item I have no idea.
\end{enumerate}

\section{Problem 3}

\subsection{Part A}

We have:

\begin{align*}
    Q_1 &\subseteq Q_4      \tag{$h = (x, y) \to (x, y)$} \\
    \\
    Q_2 &\subseteq Q_1      \tag{$h = (x, y, z, u) \to (z, u, z, u)$} \\
    Q_2 &\subseteq Q_4      \tag{$h = (x, y, z, u) \to (x, y, x, x)$} \\
    \\
    Q_3 &\subseteq Q_1      \tag{$h = (x, y, z, u) \to (x, y, x, y)$} \\
    Q_3 &\subseteq Q_2      \tag{$h = (x, y, z, u) \to (x, y, z, x)$} \\
    Q_3 &\subseteq Q_4      \tag{$h = (x, y, z, u) \to (x, y, x, x)$} \\
    \\
    Q_4 &\subseteq Q_1      \tag{$h = (x, y, z, u) \to (x, y, x, y)$} \\
\end{align*}

Note that $Q_1 \equiv Q_4$.

Homomorphisms found via brute force. \url{https://gist.github.com/tylerhou/7a67cee962214de408a9ec88334f25ce}

\subsection{Part B}

\begin{enumerate}
    \item $Q_1 \equiv Q_2$. $Q_1$ is a query that evaluates to true on a
        directed triangle. $Q_2$ is a query that evaluates to true on a
        directed triangle with some vertex $x \geq y$, where $x$ has an edge to
        $y$. It is clear that $Q_2 \subseteq Q_1$.

        To show the other direction, consider any directed triangle on three
        vertices. Let $x$ be a maximal vertex, and consider its outgoing edge
        to $y$. Either $x = y$ or $x > y$ by assumption of maximality.

    \item $Q_3 \subseteq Q_1$. This is clear.

    \item $Q_1 \not\subseteq Q_3$, and $Q_2 \not\subseteq Q_3$. $Q_3$ is true
        for all directed triangles that have a path with increasing vertex
        values. The triangle $0 \to 2 \to 1 \to 0$ is a counter example for
        both cases: it only has three paths, and none are increasing. However,
        it is a directed triangle, and there is one increasing edge ($0 \to
        2$).

    \item $Q_3 \subseteq Q_2$. If a triangle has an increasing path, it also
        has an increasing edge.
\end{enumerate}

\subsection{Part C}

$Q_1 \subseteq Q_2$. Suppose that a graph has a (possibly non-simple) cycle of
length $5$ such that the first and second vertices are not equal. We show that
it must also have a cycle of length $5$ where the first and third vertices are
not equal.

Let $a \to b \to c \to d \to e \to a$ be the cycle of length $5$ where $a \neq
b$. There are two cases. If $a \neq c$, then we are done. Otherwise, $a = c$.
Then, the graph has a cycle of length $3$: $(c =)\;a \to d \to e \to a$. Again,
there are two cases. If $a \neq e$, we have the cycle $a \to d \to e \to a \to
b \to c\;(= a)$ with the desired property. Otherwise, $a = e$ which implies
$R(a, a)$: there is a self edge from $a$ to itself. Hence, the path $a \to a
\to b \to a \to a \to a$ is a cycle of length $5$ with the desired property.

\bigskip
The other direction: $Q_2 \subseteq Q_1$. Suppose a graph has a (again possibly
non-simple) cycle of length $5$ such that the first and third vertices are not
equal.

Again, call such a cycle $a \to b \to c \to d \to e \to a$ such that $a \neq
c$. There are two cases; if $a \neq b$, we are done. Otherwise, $a = b$, so
again $a$ has a self loop, and $a$ has an edge to $c$. We can create a cycle $a
\to c \to d \to e \to a \to a$ of length $5$ where the first and second
vertices are different.


\end{document}
