\documentclass{article}
\usepackage[utf8]{inputenc}
\usepackage[headings]{fullpage}
\usepackage{algorithm2e, amsfonts, amsmath, amssymb, enumitem, fancyhdr, hyperref, mathtools, minted, MnSymbol, stmaryrd, tikz, xfrac}
\usetikzlibrary{positioning}
\graphicspath{ {.} }

\usepackage[letterpaper,
            left=2cm,
            right=2cm,
            top=3cm,
            bottom=2cm]{geometry}

\def\Name{Tyler Hou}
\def\StudentID{3033060357}
\def\Class{CS 294}
\def\Semester{Fa 2023}
\def\HomeworkNumber{1}
\def\DocumentTitle{\Class{}, \Semester{} -- HW \HomeworkNumber{}}

\newcommand{\Continued}{
    \begin{flushright}
    $\xrightarrow{\text{Continued on next page}}$
    \end{flushright}
    \newpage{}
}
\DeclarePairedDelimiter\ceil{\lceil}{\rceil}
\DeclarePairedDelimiter\floor{\lfloor}{\rfloor}
\DeclarePairedDelimiter\abs{\lvert}{\rvert}
\DeclarePairedDelimiter\linvec{\langle}{\rangle}

\def\Algorithm{\medbreak\textit{Algorithm}}
\def\Proof{\medbreak\textit{Proof of correctness}}
\def\Runtime{\medbreak\textit{Runtime analysis}}

\newcommand{\BigO}{\mathcal{O}}
\newcommand{\qed}{\hfill$\square$}

\title{\vspace{-2cm}\DocumentTitle{}}
\author{\Name{} (SID \StudentID{})}
\date{}

\pagestyle{fancy}
\fancyhf{}
\lhead{\DocumentTitle{} -- \Name{} (\StudentID{})}
\rhead{\thepage}

\begin{document}

\maketitle

\noindent\textbf{Collaborators}:
Chris Douglas,
Vivian Fang,
Altaan Han,
Shadaj Laddad,
Gabriel Matute,
Jiwon Park, \newline
Conor Power,
Ricardo Sandoval

\section{Problem 1}

\begin{enumerate}[label = {\alph*)}]
    \item See attached scan.
    \item See attached scan.
\end{enumerate}

\subsection{Part C}

(I don't know if this is correct, but this is what I came up with after about
15 hours of work that isn't just a ``trivial'' application of EF games. I can't
find anything obviously wrong with it, but it probably has a flaw...)

\bigskip

We claim (Claim 1) that every sentence $\phi$ in such a language can be written
in the form:

\newcommand{\AND}{\wedge}
\newcommand{\OR}{\vee}

\begin{align*}
    \phi &= \psi_1 \OR \psi_2 \OR \psi_3 \ldots \\
    \psi_n &= \exists x_n^1 \exists x_n^2 \ldots \exists x_n^{k_n} (\chi_n^1 \AND \chi_n^2 \AND \ldots \chi_n^t) \\
    \chi_n^t &= (x_n^i = x_n^j) \; | \; (x_n^1 \neq x_n^j) \; \text{for some $1 \leq i, j \leq k_n$. $|$ is in the meta language.} \\
\end{align*}

In such a form, we can interpret each $\psi_n$ term as a (possibly
unsatisfiable) claim on the size of the domain. To do this, draw a graph where vertices are variables and edges of type A are
equalities between variables and edges of type B are inequalities between
variables.

Next, consider all simple cycles of the graph. If there is a simple cycle where
all edges are of type A (equalities) except one edge is of type B
(inequalities), the $\psi_n$ is not satisfiable. (One example of such an
unsatisfiable sentence s $(a = b) \AND (b = c) \AND (c \neq a)$. A simple cycle
is $a = b = c \neq a$, and there is exactly one inequality in that cycle.)

Otherwise, consider the longest simple cycle using edges of just type B
(inequalities); let the number of vertices in that cycle be $d$. Then $\psi_n$
claims that the domain must have size at least $d$.

Then $\phi$ is the union of all sentences. If the union is non-empty (at least
one $\psi_i$ is satisfiable) then $\phi$ is satisfiable by any domain $D$ of
size $d_i$ for all satisfiable $\psi_i$. Otherwise, $\phi$ is not satisfiable.

\bigskip

\subsection{Proof of Claim 1}

\smallskip

Let $\phi$ be a sentence using symbols $\neg, \forall, \exists, \AND, \OR, =,$
and $\neq$. We rewrite $\phi$ to be in the above form via the following steps:

\begin{enumerate}[label = {\arabic*)}, itemsep=0mm]
    \item We push down $\neg$ into the ``leaves'' of $\phi$, eventually
        replacing $\neg(a = b)$ with $a \neq b$.
    \item We remove $\forall$'s by replacing $\forall x_1 \forall x_2 \ldots
        \alpha$ and $\exists x \forall y \alpha$ from outside in with
        equivalent expressions that do not use $\forall$.
    \item We normalize the resulting expression (that only uses $\exists, \AND,
        \OR, \neq,$ and  $=$) into the above desired form.
\end{enumerate}

\subsubsection{Step 1: Push down $\neg$}

The following rewrite rules suffice to push down $\neg$. A full proof would
induct over the distance from the topmost $\neg$ in the expression tree to its
leaves.

\begin{enumerate}[label = {\alph*)}, itemsep=0mm]
    \item $\neg \forall x \alpha \equiv \exists x \neg \alpha$
    \item $\neg \exists x \alpha \equiv \forall x \neg \alpha$
    \item $\neg (\alpha \AND \beta) \equiv \neg \alpha \OR \neg \beta$ (De Morgan's)
    \item $\neg (\alpha \OR \beta) \equiv \neg \alpha \AND \neg \beta$ (De Morgan's)
    \item $\neg (a = b) \equiv a \neq b$
    \item $\neg (a \neq b) \equiv a = b$
\end{enumerate}

Note for the inductive steps a-d, $\neg$ is pushed down by one level on every
application of the rewrite. Once $\neg$ surrounds an equality (or inequality),
we can remove it entirely by applying rule e (or f).

\subsubsection{Step 2: Replace $\forall$ with $\exists$}

After Step 1, our expression only contains $\forall, \exists, \AND, \OR, =,$
and $\neq$.

Note that if the domain only has one element, all quantifers trivially reduce
to quantifying over that single element. Hence, to check whether a query is
satisfiable by any domain with one element, we can replace all variables with a
single constant, remove all quantifiers, and see if the resulting formula (with
no variables) evaluates to $\top$ (true). If it does, because our new sentence
did not rely on any property of the element, we know that {\em all} domains
with size one satisfy the sentence. On the other hand, if the new sentence
evaluates to $\bot$ (false), then we know that {\em no} domain with size one
can satisfy the sentence.

Therefore, for the rest of the proof, we assume that we are checking whether
the sentence $\phi$ is satisfiable by some domain with at least two elements.
Under that assumpion, we claim that the following three rewrite rules hold.

\bigskip
\textbf{1. }{\boldmath $\forall x \forall y \; \theta \equiv (\exists x \; \theta[y/x]) \AND (\exists x \exists y \; (x \neq y) \AND \theta)$ }
\smallskip

$\theta[y/x]$ denotes the expression $\theta$ where all instances of $y$ have
been replaced with $x$.

\textit{Justification.} Assuming that the domain has at least two elements,
there are two cases for $x$ and $y$ when we range over both. Either they are
equal, or they are not. The first term on the right hand side handles the equal
case, and the second term handles the not equals case.


\bigskip
\textbf{2. }{\boldmath $\exists x \forall y \; \theta \equiv (\exists x \; \theta[y/x]) \AND (\exists x \exists y \; (x \neq y \AND \theta))$}
\smallskip

\textit{Justification.} For a similar reason as above. Once we fix a particular
$x$, there are two possibilities for $y$. Either it is equal to $x$, which is
handled by the first term of the left hand side. Otheriwse, it is not equal to
$x$, so it is handled by the right hand side.


\bigskip
\textbf{3. }{\boldmath $\forall x \exists y \; \theta \equiv (\exists x \; \theta[y/x]) \OR (\exists x \exists y \; (x \neq y \AND \theta))$}
\smallskip

Note that we have a disjunction instead of a conjunction on the right hand
side.

\textit{Justification.} Suppose we fix a particular $x$. Now we must choose
some $y$ to satisfy $\theta$. There are two cases: either we choose a $y$ that
is equal to $x$ (first term), or we choose a $y$ that is not equal to $x$
(second term). Because only one needs to be satisifed, the terms are joined
with an $\OR$.

By similar reasoning, we can also rewrite expressions like $\forall x ((\forall
y \; \theta) \OR (\exists z \; \rho))$. For a particular $x$ and $y$, there are
two cases (equals and not equals), and similarly for $x$ and $z$. Hence, we
have


$$
\forall x ((\forall y \; \theta) \OR (\exists z \; \rho)) \equiv \exists x \; [(\theta[y/x] \AND (\exists y \; (x \neq y \AND \theta))) \OR (\theta[z/x] \OR \exists z \; (x \neq z \AND \rho)]
$$

There are five more cases of this variety. All six are:

\begin{enumerate}[label = {\roman*)}, itemsep=0mm]
    \item $\forall x ((\forall y \; \theta) \OR (\forall z \; \rho))$
    \item $\forall x ((\forall y \; \theta) \OR (\exists z \; \rho))$
    \item $\forall x ((\exists y \; \theta) \OR (\exists z \; \rho))$
    \item $\exists x ((\forall y \; \theta) \AND (\forall z \; \rho))$
    \item $\exists x ((\forall y \; \theta) \AND (\exists z \; \rho))$
    \item $\exists x ((\exists y \; \theta) \AND (\exists z \; \rho))$
\end{enumerate}

\bigskip

We leave the other five equivalences as an exercise to the grader (although we
note that their equivalences are the results of mechanically applying rewrites
1, 2, and 3 from above).


Finally, for cases where we have expressions of the form $\forall x \; (\theta
\AND \rho)$ and $\exists y \; (\theta \OR \rho)$ we can simply distribute:

\begin{enumerate}[label = {-}, itemsep=0mm]
  \item $\forall x \; (\theta \AND \rho) \equiv (\forall x \; \theta) \AND (\forall x \; \rho)$
  \item $\exists x \; (\theta \OR \rho) \equiv (\exists x \; \theta) \OR (\exists x \; \rho)$
\end{enumerate}

\subsubsection{Step 3}

At this point, our expression only contains $\exists, \AND, \OR, =,$ and
$\neq$.

We apply the following two rewrite rules repeatedly to push down $\AND$'s and
lift up $\OR$'s. The first is a repeat of the $\exists$ distributive rule
above.

\begin{enumerate}[label = {\alph*)}]
    \item Lift Up $\OR$: $\exists x \; (\theta \OR \rho) \equiv (\exists x \; \theta) \OR (\exists x \; \rho)$
    \item Push Down $\AND$: $\exists x \; [(\theta \OR \rho) \AND \sigma] \equiv \exists x \; [(\theta \AND \sigma) \OR (\rho \AND \sigma)]
        \\ \hspace*{47mm} \equiv (\exists x \; (\theta \AND \sigma)) \OR (\exists x \; (\theta \AND \sigma))$
\end{enumerate}

Finally, we note that all rewrites above increase the size of $\phi$ by a most
a constant multiple, and each rewrite removes at least one unwanted term.
Hence, the final query is finite in size, and can be interpreted as described
in 1.1 in finite time.

It is clear that in this case that even if the structure is infinite, we can
still convert $\phi$ into the desired form in finite time, so we can decide
$\texttt{SAT}(\phi)$ for infinite structures in finite time. \qed

\subsection{Part D}

We repeat the above, except for Step 1, we push down $\neg$ to until it
modifies a relation (i.e. $\neg P(x)$ is allowed). The remaining rewrite rules
apply normally, the only difference is instead of having no $\neg$ symbol, we
allow $\neg$ to modify relations.

The final rewritten $\phi$ has the same definition as 1.1, except we also allow
relations and their negations to appear in $\chi_n^t$ (changes in
\textbf{bold}):

\begin{align*}
    \phi &= \psi_1 \OR \psi_2 \OR \psi_3 \ldots \\
    \psi_n &= \exists x_n^1 \exists x_n^2 \ldots \exists x_n^{k_n} (\chi_n^1 \AND \chi_n^2 \AND \ldots \chi_n^t) \\
    \chi_n^t &= (x_n^i = x_n^j) \; | \; (x_n^1 \neq x_n^j) \mathbf{\; | \; P_t(x_n^i) \; | \; \neg P_t(x_n^i)}
\end{align*}

Finite satisfiability for each $\psi_n$ (and thus for $\phi$) is decidable.
First, we use 1.1 to see whether the constraints on the domain's size can be
satisifed. If the constraints cannot be satisfied, we conclude that $\psi_n$ is
not satisfiable.

Otherwise, let the \textit{group} of an element $d \in D$ be the tuple $G(d) =
(A_1(d), A_2(d), \ldots, A_t(d))$ for $1 \leq t \leq T$ ($T$ is the number of
unary relations) where $A_i = 1$ if $d \in P_x$; else $A_i = 0$. Partition
elements of $D$ into their groups. Note that for the purpose of satisfying
$\psi_n$, elements in each group are symmetric.

Hence, we can brute force search, assigning $x_1$ to some element of each group
in order. For a fixed $x_1$, we assign $x_2$ to some element of the remaining
groups; then we assign $x_3$. In other words, we try all possible assignments
of $x_i$ to groups. If some assignment satisfies $\psi_n$, then we conclude
that it is satisfiable, otherwise, we conclude that it is not. This can clearly
be done in finite time (it is bounded by $\BigO(k_n^{k_n})$).

Finally, we conclude $\phi$ is satisfiable if any only if some $\psi_i$ is
satisfiable.

\subsubsection{General Satisfiability}

Then, suppose that $\phi$ is generally satisfiable; we show that it is finitely
satisfiable. If $\phi$ is satisfiable, then its rewritten version is also
satisfiable, so some $\psi_n$ is satisfiable.

Thus, we can enumerate elements $d \in D$ (where $D$ is possibly infinite in
size), sorting elements into their groups. After we process each element $d$,
we attempt to satisfy all $\psi_i$ using the above procedure. If we
successfully satisfy some $\psi_i$, we conclude $\phi$ is satisfiable;
otherwise, we continue.

(This feels like recursively enumerable, and not general satisfiability...? If
I had some magic procedure to partition $D$ into groups (of possibly infinite
size), that would make this much easier.)

\subsubsection{Trakhtenbrot’s theorem}

This does not contradict Trakhtenbrot’s theorem because Trakhtenbrot’s theorem
requires at least one binary relation. We only have unary relations in this
structure.

\section{Problem 2}

See attached scan.


\end{document}
